\raghav{TODO: learn a word that's not ``approach''}
A naive approach to vectorizing arbitrary arithmetic circuits looks a lot like SLP: We start by serializing the circuit into a sequence of scalar three-address code.
At each step, we look at all available scalar instructions (i.e. instructions whose operands have all already been scheduled), pick the largest set with the same operation, and schedule them together.
The problem with this naive approach is it makes no guarantees about values being computed and used on the same lane; in other words, it incurs arbitrarily many shuffles to make the computation line up. 
Unlike in normal vectorization, where applying arbitrary permutations to the lanes is relatively cheap, in FHE we are only allowed to rotate the entire vector by a fixed number of slots, and this rotation operation is expensive.
Hence, the cost of bookkeeping quickly outweighs whatever benefit we might get from vectorization, making this approach not worth it.

The fundamental problem that makes vectorization in an FHE setting very hard is the high cost of moving data between vector lanes. 
Data movement is actually impossible to avoid: in an arithmetic circuit that computes a single expression, all the intermediate nodes eventually feed into the root, meaning that without data movement every instruction must be on the same line; in other words, no vectorization can take place.

Here we have two extremes: on one end of the spectrum is the SLP approach, which packs aggressively without considering data movement; the other end avoids data movement entirely, precluding any vectorization within individual expressions.
The key idea of our approach is to balance between these two extremes, by finding highly vectorizable subexpressions and evaluating them together with each one on its own lane.

\subsection{Selecting Vectorizable Subexpressions} \raghav{I need annotations on this entire subsection of which things don't make sense and need to be rewritten, because I am certain there are several.}
\subsubsection*{Measuring structural similarity between subtrees}
The structural similarity of two expression trees is used as a proxy for their vectorizability; in particular, we want to measure how well the instruction sequences corresponding to two arbitrary expression trees line up.
Given an expression tree, we can inductively compute the structural similarity between every pair of independent subtrees (two trees are independent if neither root is an ancestor of the other) as follows:
The base case is comparing any tree to a leaf node (a single variable with no operations), in which case the similarity is $0$, as there are no operations that can be packed together.

In the inductive case of two nontrivial trees, there are six ways they could be lined up: \raghav{TODO: diagram}
Either root could be lined up with either child of the other tree in four ways. 
Or, the two roots could line up, giving two ways for their children to line up (left with left and right with right, or left with right and right with left).
In the first four cases, the similarity score of the two trees is the similarity score of the alignment. 
In the last two cases, the similarity score of the two trees is the sum of the scores of the aligned children, plus a 1 if the roots have the same operation.
The final similarity score of the two trees is the maximum possible score out of all six cases. 
\raghav{This makes no sense even to me, and I was looking at the code when I wrote this.}
\subsubsection*{Choosing maximally vectorizable sets}\raghav{TODO: add a running example $(ab+c)(xy+z)$}
Not every pair of subexpressions is eligible to be vectorized together, since they have to be independent of each other.
This data can be encoded in an undirected {\em vectorizability graph}, where each subexpression is a vertex, and there is an edge between two vertices if the corresponding subexpressions are independent (and thus vectorizable).
Cliques in this graph correspond exactly to sets of expressions that can all be vectorized together. 
We can further label each edge with the similarity score of the two subexpressions it connects.
Since each similarity score roughly corresponds to the number of instructions that could be packed together, the total weight of a clique represents the total number of operations we save by vectorizing together all the subexpressions contained within. 
By simply subtracting a fixed amount from the weight of each edge we can penalize large cliques which would incur a much higher rotation overhead, unless the vectorizability of the clique is high enough to be worth it.
The problem of finding a set of subexpressions to vectorize together reduces to finding a maximal weight clique in this graph, which is easily packaged off to an SMT solver.

A single round of this technique greedily selects the optimal ``breakpoints'' up to which to vectorize before inserting rotations, so we need to iterate until the entire program has been scheduled.
Once a set of breakpoints is selected \raghav{Yeah, this is the first time I'm officially using the word breakpoint so far, I should introduce it earlier. Terminology is {\em hard}.}, they need to be ``quotiented out'' (i.e. each subexpression needs to be replaced by a single leaf node in the expression tree).
This has two effects on the vectorizability graph: First, all the nodes that appear in the quotient need to be eliminated (including the {\em sub}-subexpressions of the chosen subexpressions), since they are no longer eligible to be vectorized.
However, removing these subexpressions also affects the similarities of their ancestors, since they now have fewer operations that can be packed together.
These changes must be reflected by updating the weights of the vectorizability graph.
To avoid having to recompute all the similarities each time, we store for each edge the list of operations it could pack together; the edge weight can be recovered as the sum of the costs of all of these (in the simplest case, the length).
Each time a node is quotiented out, it is removed from each list that contains it; doing so automatically updates all the edge weights to account for its removal.
Now, we can repeat the above process of finding a maximal clique and vectorizing it, until eventually all nontrivial cliques have a negative total weight, meaning that there are no more subexpressions that are vectorizable enough to make the rotation costs worth it, so we just emit scalar code for the rest of the program.

\subsection{Vector Scheduling}
Given a set of subexpressions to compute in a single phase of the program, we can align the instructions between subexpressions to actually produce a vector schedule.
Aligning trees is more complicated than a simple sequence alignment, because at each step the number of available children to align roughly doubles, meaning that the total number of subproblems to solve is exponential in the depth instead of linear. 
\raghav{Is this a good enough explanation for why sequence alignment doesn't work here?}

Instead of wrangling so many subproblems, we can easily pass this off to an SMT solver.
The encoding consists of a variable for each instruction representing when it gets scheduled, as well as constraints that require two instructions scheduled at the same time to have the same operation, and constraints that require the two dependencies of each instruction to be scheduled before it.
Finally, to speed up the search for a solution, we place a bound on the maximum number of slots in the schedule (i.e. the latest time an instruction can be scheduled).
This bound is initially very loose, but once a schedule is produced we iteratively tighten it until the solver returns UNSAT, meaning no smaller schedule could be found. 
To prevent compilation from taking too long, we also set a timeout after which the solver simply returns the best available schedule.
Of course, this means that the vector schedule is no longer guaranteed to be optimal, but this timeout can be adjusted, allowing for a tradeoff between compilation time and optimality.

\subsection{Lane Placement} \raghav{``Correspond'' is also becoming an overused word}
Scheduling the program in the way described above amounts to splitting it into a number of {\em phases}, where each phase consists of a set of subexpressions to compute.
For a program that is split into $k$ phases, we can represent these dependencies in a $k$-partite graph, where each partition corresponds to a phase and each vertex in a partition corresponds to a subexpression computed in that phase.
Placing expressions to lanes corresponds to assigning an integer to each vertex in this dependency graph such that no two vertices in the same partition get mapped to the same integer.
Unfortunately, a naive approach to this can have very poor results, leading to a single vector needing several rotations to line each slot up with where it gets used.
Instead, we try to assign lanes in a way that minimizes the number of {\em distinct} rotations required for each vector (for example, if multiple pieces of data on the same vector are produced two lanes away from where they are consumed, we can rotate the vector a single time to get them both to line up).
In other words, given a $k$-partite graph with an integer associated to each vertex, we can assign to each edge the difference of its two endpoints, representing the rotation. 
To minimize the distinct numbers we have to assign to the edges, we'd like to be able to go backwards: in other words, we want to assign as few numbers as possible to all the edges such that ``integrating'' them gives a consistent assignment of numbers to the vertices.

This consistency constraint can be reworded as follows: For every path through the $k$-partite graph that starts and ends on the same partition, the directed sum of the edge weights along that path must be 0 if the path is a cycle, or nonzero otherwise.
(Notice that this automatically enforces the condition that two different paths between a pair of vertices on the same partition must add up to the same value, since concatenating one path with the reversal of the other produces a cycle, which must sum to 0).
\raghav{If only I could figure out a neat and tidy way to describe this whole process mathematically\dots}

Iterating over all the paths in the graph, we end up with a bunch of linear {\em path relations} that look like $$R_k(x_{i_1}, \dots, x_{i_{n_k}})$$ where initially all of the $x_i$s are unassigned.
To assign a particular $x_i$, if it is part of a relation where it is the only unassigned variable, its value is fixed to be one that satisfies the relation; otherwise, it can be assigned freely.
This data forms a hypergraph, where the vertices correspond to edges in the $k$-partite dependency graph, and each path relation becomes a hyperedge connecting the associated vertices.
In order to find a consistent assignment of rotation values to edges we need to color this hypergraph in such a way that the last vertex to be colored on any hyperedge must be given a distinct color from the rest of the vertices on the hyperedge.
Obviously, to minimize the number of distinct rotations needed we would like to use as few colors as possible.
This is relatively straightforward to do: At each step we pick the vertex that is part of as few ``free'' hyperedges as possible and assign it the next available color, until all the vertices have been colored.

Now, all that remains is getting actual integer values for each color. 
This can be formulated as a simple integer program, with a variable $c_i$ for each color, a variable  $v_j$ for each subexpression, and a constraint for each edge $(v, v')$ in the dependency graph colored $c$ to assert $v - v' == c$.

\subsection{Code Generation}
